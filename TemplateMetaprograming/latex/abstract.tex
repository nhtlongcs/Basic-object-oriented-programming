\begin{Abstract}
% \section{Tóm tắt báo cáo}
\begin{center}

\Large
\vspace{0.9cm}
\textbf{Tóm tắt}
\vspace{0.9cm}
\end{center}
\textbf{Metaprogramming} là cách viết mã nguồn bằng cách xây dựng các \textbf{template}. Từ các \textbf{template} được quy định trình biên dịch sẽ xem cách thức gọi các \textbf{template} và sinh ra mã nguồn thực dựa trên quy trình gọi và quá trình thực thi. \textbf{Metaprogramming} được sử dụng để nâng cao hiệu suất bằng cách thực hiện tối ưu hóa \textbf{run-time} bằng cách đưa các tác vụ có thể tính toán tại \textbf{compile-time} và lưu trữ, chẳng hạn như làm việc gì đó một lần tại \textbf{compile-time} thay vì mỗi khi chương trình được chạy - ví dụ, bằng cách yêu cầu trình biên dịch bỏ các vòng lặp để loại bỏ các bước và giảm số vòng lặp bất cứ khi nào chương trình được thực thi.

\vspace{0.2cm}

Phần báo cáo được chia ra làm \textbf{5 mục chính}, bao gồm giới thiệu cú pháp và nền tảng cơ bản với \textbf{template}, kế đến sẽ đi vào các khái niệm và nguyên tắc về thiết kế, cụ thể là sự linh động của \textbf{tính đa hình} được trong việc thiết kế các thư viện. Tiếp tục thảo luận sâu hơn về các nguyên tắc và tính rành mạch của ta sẽ đi sâu hơn vào khái niệm \textbf{concept}. Cuối cùng là thực nghiệm với các ví dụ và đưa ra các nhận định, thảo luận sau quá trình nghiên cứu về \textbf{metaprogramming}.

\vspace{0.2cm}

Trong bài sẽ sử dụng các thuật ngữ chuyên môn, vì để đảm bảo toàn vẹn nội dung, nhóm quyết định sẽ không sử dụng hoàn toàn ngôn ngữ thuần Việt.\textbf{ Bảng thuật ngữ được ghi chú tại phụ lục \ref{tab:notion}}

\end{Abstract}
