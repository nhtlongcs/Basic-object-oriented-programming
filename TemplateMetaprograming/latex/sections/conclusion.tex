\section{Kết luận}
\subsection{So sánh ưu nhược điểm}
Xét về \textbf{ưu điểm}, sử dụng \textbf{template metaprograming} sẽ nâng cao tính tổng quát của chương trình bằng việc chỉ tập trung vào kiến trúc, và đẩy việc cài đặt các hàm cụ thể cho trình biên dịch khi người dùng yêu cầu. Nhờ đó có thể giảm kích thước mã và dễ bảo trì hơn. Ngoài ra việc mở rộng chương trình và \textbf{tránh sự trùng lặp mã nguồn} cũng được cải thiện đáng kể. Việc sử dụng type-checking, type-safe giúp chúng ta kiểm soát được lỗi, \textbf{giảm thiểu lỗi do chương trình có thể tự đánh giá trong compile-time}. Tiếp đến \textbf{hiệu suất, thời gian chạy cũng được nâng cao} bằng việc tính toán hỗ trợ trong compile-time. Các thư viện lập trình theo \textbf{function-oriented} như \textbf{Boost} đã được công nhận và sử dụng rộng rãi vì tốc độ thực thi nhanh và tính linh động của nó.


Tuy vậy, không thể phủ nhận việc sử dụng \textbf{template metaprograming} đi ngược lại với tính cách hay hướng viết mã nguồn kỉ luật của nhiều lập trình viên vì sự\textbf{ khó đọc}. Mã nguồn của một chương trình sử dụng hoàn toàn template metaprograming rất\textbf{ cồng kềnh} và cần phải biết được các ngữ cảnh hay các thiết kế phổ biến để có thể sử dụng. Các đối tượng được xem xét trong template metaprograming là \textbf{immutable}, vì vậy thiết kế giải pháp cho những tác vụ đơn giản như trong hướng đối tượng sẽ \textbf{khó hơn gấp nhiều lần} trong template meta programming, có thể kể đến như chuyển từ lập trình tuần tự sang lập trình đệ quy. Ngoài ra nếu không tuân thủ các quy định đóng góp mã nguồn, template meta programing sẽ thông \textbf{báo lỗi khó hiểu để có thể kiểm soát về các kiểu dữ liệu}. Hơn nữa với các trình biên dịch lỗi thời, các cú pháp trong trường phái này có thể \textbf{không tương thích với nhiều trình biên dịch}.

\subsection{Tóm tắt và thảo luận}
\textbf{Tổng kết}, qua quá trình nghiên cứu, nhóm đã tìm hiểu về công dụng của Metaprograming sử dụng \textbf{Template trong C++ }và các ý tưởng sơ khai. Trọng điểm nhất là việc tham số hoá kiểu dữ liệu và các nguyên tắc thiết kế. Nhóm đã đưa ra các ví dụ, triển khai và phát triển. Các thực nghiệm và dẫn chứng thiết thực về tốc độ và hiệu năng cũng được khảo sát và trình bày. 

Song vẫn còn nhiều vấn đề tồn đọng và các trường hợp sử dụng kết hợp linh hoạt các thành phần trong metaprograming chưa được đề cập đến. Hay về các nội dung đề cập, nhóm chỉ tiến hành giới thiệu về \textbf{polymorphic behavior} của templates chứ không đề cập về static polymorphism. Điều này có thể đề cập đến \textbf{CRTP}. \textbf{CRTP} là viết tắt của \textbf{Curily Recurring Template Pattern} và có nghĩa là một kỹ thuật trong C ++, trong đó kế thừa một lớp \textbf{Derived} từ một lớp mẫu \textbf{Based} và \textbf{Derived} sử dụng \textbf{Base} như một tham số mẫu.